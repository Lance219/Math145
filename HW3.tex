\documentclass[12pt,letterpaper]{article}

%\usepackage{url}
\usepackage{fancyhdr,fancybox,comment}
\usepackage{amstext}
\usepackage{amsmath,amssymb,xcolor,enumerate}
\usepackage{rotating}
%\usepackage{multicol}
%\usepackage{pictexwd,pictexplus}
\usepackage{pictexwd}



\setlength{\textwidth}{7in}
\setlength{\evensidemargin}{-0.25in}
\setlength{\oddsidemargin}{-0.25in}
\setlength{\parskip}{.3\baselineskip}
\setlength{\topmargin}{-1.0in}
%\setlength{\topmargin}{-0.2in}
%\setlength{\topmargin}{-0in}
\setlength{\textheight}{9.5in}

\newcommand{\ds}{\displaystyle}
\newcommand{\add}[1]{{\color{blue} #1}}
\newcommand{\alert}[1]{{\color{red} #1}}
\newcommand{\pnt}[1]{\textbf{[#1]}}
\newcommand{\q}{\pnt{5} }
\newcommand{\hint}[1]{\noindent{\color{purple} Hint: #1}}
\renewcommand{\mark}[1]{\noindent{\color{red} #1}}
\def\R{{\mathbb R}}
\def\Z{{\mathbb Z}}
\def\N{{\mathbb N}}
\def\Q{{\mathbb Q}}
\def\C{{\mathbb C}}
\def\F{{\mathbb F}}
\def\lcm{{\rm lcm}}
\def\prob#1{\vskip13pt\noindent\llap{{\bf #1:\ }}}
\font\Bigrm=cmr12 scaled\magstep1

%               Problem and Part
\newcounter{problemnumber}
\newcounter{partnumber}
\newcommand{\Problem}{\stepcounter{problemnumber}\setcounter{partnumber}{0}\item[\fbox{\parbox{.18in}{\hfil\theproblemnumber\hfil}}]}
\newcommand{\InProblem}[1]{\stepcounter{problemnumber}\setcounter{partnumber}{0}\fbox{\parbox{.18in}{\hfil\theproblemnumber\hfil}}\ \ \parbox[t]{1.9in}{#1}}
\newcommand{\InBigProblem}[1]{\stepcounter{problemnumber}\setcounter{partnumber}{0}\fbox{\parbox{.18in}{\hfil\theproblemnumber\hfil}}\ \ \parbox[t]{2.75in}{#1}}
\newcommand{\Part}{\stepcounter{partnumber}\item[(\alph{partnumber})]}
\newcommand{\InPart}[1]{\stepcounter{partnumber}(\alph{partnumber})\ \ \parbox[t]{2.25in}{#1}}
\newcommand{\InSmallPart}[1]{\stepcounter{partnumber}(\alph{partnumber})\ \ \parbox[t]{1.25in}{#1}}

%% For Answerbox:
\newcounter{answernumber}
\newcommand{\Ans}{%
  \stepcounter{answernumber}
  \fbox{\parbox{.18in}{\hfil\theanswernumber\hfil}}
}



\pagestyle{empty}

\begin{document}
%\thispagestyle{fancy}
%%
\noindent{\Bigrm HW 3 due Friday September 29}



\prob1\pnt{4} Let $m\leq n$ be positive integers. Prove that $\displaystyle\frac{n}{\gcd(n,m)}\mid\binom{n}{m}.$


\vspace{5pt}


\prob2\q Let $h(x) = b_nx^n + \cdots + b_1x + b_0$ be a non-constant polynomial with integer coefficients. Prove that the set $\{p\text{ prime}\colon p\mid h(m)\text{ for some }m\in\Z\text{ such that }h(m)\neq 0\}$
    is infinite.

\noindent This is also known as Schur's Theorem.

\hint{Try the cases $b_0 = 0$ and $b_0 = 1$ first. Then think about how you can reduce to these two cases.}

\vspace{5pt}


\prob3\q Let $\mu:\N\rightarrow\{-1,0,1\}$ denote the Mobius function defined by
$$\mu(n) = \begin{cases}(-1)^{d(n)}&\mbox{if }n\mbox{ is squarefree}\\
0&\mbox{otherwise}\end{cases}$$
where $d(n)$ denotes the number of distinct prime divisors of $n$. Prove:
\begin{enumerate}[(a)]
    \item \pnt{3} $\displaystyle \sum_{d\mid n}\mu(d) = \begin{cases}
    1&\mbox{if }n=1,\\
    0&\mbox{if }n>1.
\end{cases}$

    \hint{How many divisors of $n$ are products of $j$ distinct prime divisors of $n$?}
    \item \pnt{2} If $f,g:\N\rightarrow\C$ satisfy $\displaystyle f(n) = \sum_{d\mid n}g(d)$, then $\displaystyle g(n) = \sum_{d\mid n}\mu(d)f(n/d).$

    \hint{If you aren't sure how to manipulate the sums, try some small values of $n$.}
\end{enumerate}


\vspace{5pt}


\prob4\pnt{6} Prove the following results about the $n$-th cyclotomic polynomial:
\begin{enumerate}[(a)]
    \item \pnt{2} For any $n\geq 1$, $\displaystyle\Phi_n(x) = \prod_{d\mid n}(x^{n/d} - 1)^{\mu(d)}.$

    \hint{The result of Problem 3(b) can't be used, but its proof might.}
    \item \pnt{2} For any $n\geq 2$, $\Phi_n(x)$ is reciprocal, namely $\Phi_n(1/x)x^{\phi(n)} = \Phi_n(x)$.
    \item \pnt{2} For any $n\geq 2$, $\log(\Phi_n(1)) = \Lambda(n)$, the von Mangoldt function evaluated at $n$.

    \hint{Compute $\ds\prod_{d\mid n,d>1}\Phi_d(1).$}
\end{enumerate}

\vspace{5pt}


\prob5\q Recall the Fermat numbers $F_n = 2^{2^n}+1$ for non-negative integers $n$. Prove that any prime divisor $p$ of $F_n$ is congruent to $1$ mod $2^{n+1}$.

\hint{What is $o_p(2)$ equal to?}

\noindent After we learn about quadratic reciprocity, we will be able to prove the stronger result that the prime divisors of $F_n$ are congruent to $1$ mod $2^{n+2}$ for $n\geq 2$.


\vspace{5pt}





\end{document}


%%% Local Variables:
%%% mode: latex
%%% TeX-master: t
%%% End:
