\documentclass[12pt,letterpaper]{article}

%\usepackage{url}
\usepackage{fancyhdr,fancybox,comment}
\usepackage{amstext}
\usepackage{amsmath,amssymb,xcolor,enumerate}
\usepackage{rotating}
%\usepackage{multicol}
%\usepackage{pictexwd,pictexplus}
\usepackage{pictexwd}



\setlength{\textwidth}{7in}
\setlength{\evensidemargin}{-0.25in}
\setlength{\oddsidemargin}{-0.25in}
\setlength{\parskip}{.4\baselineskip}
\setlength{\topmargin}{-1.0in}
%\setlength{\topmargin}{-0.2in}
%\setlength{\topmargin}{-0in}
\setlength{\textheight}{9.5in}

\newcommand{\ds}{\displaystyle}
\newcommand{\add}[1]{{\color{blue} #1}}
\newcommand{\alert}[1]{{\color{red} #1}}
\newcommand{\pnt}[1]{\textbf{[#1]}}
\newcommand{\q}{\pnt{5} }
\newcommand{\hint}[1]{\noindent{\color{purple} Hint: #1}}
\def\R{{\mathbb R}}
\def\Z{{\mathbb Z}}
\def\N{{\mathbb N}}
\def\Q{{\mathbb Q}}
\def\C{{\mathbb C}}
\def\F{{\mathbb F}}
\def\lcm{{\rm lcm}}
\def\prob#1{\vskip13pt\noindent\llap{{\bf #1:\ }}}
\font\Bigrm=cmr12 scaled\magstep1

%               Problem and Part
\newcounter{problemnumber}
\newcounter{partnumber}
\newcommand{\Problem}{\stepcounter{problemnumber}\setcounter{partnumber}{0}\item[\fbox{\parbox{.18in}{\hfil\theproblemnumber\hfil}}]}
\newcommand{\InProblem}[1]{\stepcounter{problemnumber}\setcounter{partnumber}{0}\fbox{\parbox{.18in}{\hfil\theproblemnumber\hfil}}\ \ \parbox[t]{1.9in}{#1}}
\newcommand{\InBigProblem}[1]{\stepcounter{problemnumber}\setcounter{partnumber}{0}\fbox{\parbox{.18in}{\hfil\theproblemnumber\hfil}}\ \ \parbox[t]{2.75in}{#1}}
\newcommand{\Part}{\stepcounter{partnumber}\item[(\alph{partnumber})]}
\newcommand{\InPart}[1]{\stepcounter{partnumber}(\alph{partnumber})\ \ \parbox[t]{2.25in}{#1}}
\newcommand{\InSmallPart}[1]{\stepcounter{partnumber}(\alph{partnumber})\ \ \parbox[t]{1.25in}{#1}}

%% For Answerbox:
\newcounter{answernumber}
\newcommand{\Ans}{%
  \stepcounter{answernumber}
  \fbox{\parbox{.18in}{\hfil\theanswernumber\hfil}}
}



\pagestyle{empty}

\begin{document}
%\thispagestyle{fancy}
%%
\noindent{\Bigrm HW 1 due Friday September 15}


\prob1\q Let $a,b$ be positive integers with $\gcd(a,b) = 1$. 
\begin{enumerate}[(a)]
    \item \pnt{2} Prove that $ab - a - b$ cannot be written as $ax + by$ where $x,y$ are non-negative integers.

    \item \pnt{3} Prove that for any integer $n > ab - a - b$, there exist non-negative integers $x,y$ such that $n = ax + by$.

    \hint{The best chance for a solution is to take $x$ to be as small as possible and hope that the corresponding $y$ is non-negative.}

    \hint{In order for some non-negative integer $x$ to be part of a solution, we need there to exist an integer $y$ so that $n - ax = by$. This suggests considering the set $S = \{x\in\Z\colon b\mid n - ax, x\geq 0\}$.}
\end{enumerate}



\prob2\pnt{5} Let $\Z_{\geq0}$ denote the set of non-negative integers. Let $f:\Z_{\geq0}\times\Z_{\geq0}\rightarrow\Z_{\geq0}$ be a function such that $f(b,a) = f(a,b) = f(b-a,a)$ for any integers $0\leq a\leq b$. Prove that $f(a,b) = f(\gcd(a,b),0)$ for any $a,b\in\Z_{\geq0}$.

\hint{This screams induction. Induct on something that measures the ``size of $a$ and $b$''. This question is basically asking you to prove that the Euclidean Algorithm works, using induction. If you prove it without induction, make sure your wordings are precise enough.}

\prob3\pnt{5} Let $a,m,n$ be positive integers. Prove that $\gcd(a^m-1, a^n-1) = a^{\gcd(m,n)}-1.$

\hint{Define a convenient $f:\Z_{\geq0}\times\Z_{\geq0}\rightarrow\Z_{\geq0}$ and use Problem 2.}


\prob4\pnt{5} Let $x,y,z$ be positive integers such that $\gcd(x,y,z) = 1$ and $\displaystyle \frac{1}{x} + \frac{1}{y} = \frac{1}{z}$. Prove that $x+y$ is a perfect square.

\hint{A positive integer $n$ is a perfect square if and only if $\nu_p(n)$ is even for every prime $p$.}



\prob5\pnt{5} Let $a_1,a_2,a_3,\ldots$ be a sequence of positive integers such that $\gcd(a_m,a_n) = \gcd(m,n)$ for any $m,n\in\N$ that are distinct. Prove that $a_m = m$ for any $m\in\N$.

\hint{Prove $m\mid a_m$ first.}



\end{document}


%%% Local Variables:
%%% mode: latex
%%% TeX-master: t
%%% End:
