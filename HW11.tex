\documentclass[12pt,letterpaper]{article}

%\usepackage{url}
\usepackage{fancyhdr,fancybox,comment}
\usepackage{amstext}
\usepackage{amsmath,amssymb,xcolor,enumerate}
\usepackage{rotating}
%\usepackage{multicol}
%\usepackage{pictexwd,pictexplus}
\usepackage{pictexwd}



\setlength{\textwidth}{7in}
\setlength{\evensidemargin}{-0.25in}
\setlength{\oddsidemargin}{-0.25in}
\setlength{\parskip}{.3\baselineskip}
\setlength{\topmargin}{-1.0in}
%\setlength{\topmargin}{-0.2in}
%\setlength{\topmargin}{-0in}
\setlength{\textheight}{9.5in}

\newcommand{\ds}{\displaystyle}
\newcommand{\add}[1]{{\color{blue} #1}}
\newcommand{\alert}[1]{{\color{red} #1}}
\newcommand{\pnt}[1]{\textbf{[#1]}}
\newcommand{\q}{\pnt{5} }
\newcommand{\hint}[1]{\noindent{\color{purple} Hint: #1}}
\renewcommand{\mark}[1]{\noindent{\color{red} #1}}
\newcommand{\tot}[2]{\noindent\textsc{\footnotesize This homework is marked out of \textbf{#1} and graded out of \textbf{#2}}.}
\newcommand{\ttt}{\tot{25}{20}}
\renewcommand{\lg}[2]{\Big(\frac{#1}{#2}\Big)}
\newcommand{\leg}[2]{\left(\frac{#1}{#2}\right)}
\def\R{{\mathbb R}}
\def\Z{{\mathbb Z}}
\def\N{{\mathbb N}}
\def\Q{{\mathbb Q}}
\def\C{{\mathbb C}}
\def\F{{\mathbb F}}
\def\lcm{{\rm lcm}}
\def\prob#1{\vskip13pt\noindent\llap{{\bf #1:\ }}}
\font\Bigrm=cmr12 scaled\magstep1

%               Problem and Part
\newcounter{problemnumber}
\newcounter{partnumber}
\newcommand{\Problem}{\stepcounter{problemnumber}\setcounter{partnumber}{0}\item[\fbox{\parbox{.18in}{\hfil\theproblemnumber\hfil}}]}
\newcommand{\InProblem}[1]{\stepcounter{problemnumber}\setcounter{partnumber}{0}\fbox{\parbox{.18in}{\hfil\theproblemnumber\hfil}}\ \ \parbox[t]{1.9in}{#1}}
\newcommand{\InBigProblem}[1]{\stepcounter{problemnumber}\setcounter{partnumber}{0}\fbox{\parbox{.18in}{\hfil\theproblemnumber\hfil}}\ \ \parbox[t]{2.75in}{#1}}
\newcommand{\Part}{\stepcounter{partnumber}\item[(\alph{partnumber})]}
\newcommand{\InPart}[1]{\stepcounter{partnumber}(\alph{partnumber})\ \ \parbox[t]{2.25in}{#1}}
\newcommand{\InSmallPart}[1]{\stepcounter{partnumber}(\alph{partnumber})\ \ \parbox[t]{1.25in}{#1}}

%% For Answerbox:
\newcounter{answernumber}
\newcommand{\Ans}{%
  \stepcounter{answernumber}
  \fbox{\parbox{.18in}{\hfil\theanswernumber\hfil}}
}



\pagestyle{empty}

\begin{document}
%\thispagestyle{fancy}
%%
\noindent{\Bigrm HW 11 due Wednesday December 6}

%\ttt

\prob1\pnt{4} Suppose $k$ is an integer such that $p = 6k+1$, $q = 12k+1$ and $r = 18k+1$ are all primes. Prove that $pqr$ is a Carmichael number.

\vspace{5pt}

\prob2\q Let $p,q$ be distinct odd primes and let $n = pq$. Prove that $n$ is not a Carmichael number.

\hint{Find $a$ so that $a^{pq-1}\not\equiv1\pmod{pq}$ but arranging it to be $1$ mod $q$ but not $1$ mod $p$.}

\vspace{5pt}

\prob3\q Let $p>5$ be a prime. Let $q = \displaystyle\frac{4^p+1}{5}.$
\begin{enumerate}[(a)]
    \item \pnt{2} Prove that $q$ is an integer and is composite.
    \item \pnt{3} Prove that $2^{(q-1)/2} \equiv -1\pmod{q}$.

    \hint{The only thing we know mod $q$ is $4^p$.}
\end{enumerate}

\vspace{5pt}

\prob4\q Prove that for any $n\in\N$, the Fermat number $F_n = 2^{2^n} + 1$ is prime if and only if $3^{(F_n - 1)/2}\equiv -1\pmod{F_n}$.

\hint{($\Rightarrow$) $\ds 3^{(F_n-1)/2} \equiv \lg{3}{F_n}\pmod{F_n}$. ($\Leftarrow$) What could $o_{F_n}(3)$ be?}

\vspace{5pt}

\prob5\pnt{6} Let $n = p_1^{k_1}\cdots p_r^{k_r}$ be an integer where $p_1,\ldots,p_r$ are distinct odd primes and where $r\geq 2$ and $k_i \geq 1$. For $i=  1,\ldots, r$, write $p_i - 1=  2^{e_i}m_i$ where $e_i=\nu_2(p_i-1)$ and $m_i$ is odd. Let $e = \min\{e_1,\ldots,e_r\}$.
\begin{enumerate}[(a)]
    \item \pnt{2} Let $b$ be an integer coprime to $n$. Let $\alpha_i = o_{p_i^{k_i}}(b)$ for every $i=1,\ldots,r$. Suppose $\nu_2(\alpha_1),\ldots,\nu_2(\alpha_r)$ are not all equal. Prove that $o_n(b)$ is even and $b^{o_n(b)/2}\not\equiv -1\pmod{n}$.

    \hint{$o_n(b) = \lcm(\alpha_1,\ldots,\alpha_r).$}
    \item \pnt{1} Prove that if $\nu_2(\alpha_1),\ldots,\nu_2(\alpha_r)$ are all equal, then $o_n(b)$ is odd, or $o_n(b)$ is even and $b^{o_n(b)/2}\equiv -1\pmod{n}$.
    \item \pnt{3} Prove that the number of $b\in(\Z/n\Z)^\times$ such that $o_n(b)$ is odd, or $o_n(b)$ is even and $b^{o_n(b)/2}\equiv -1\pmod{n}$ is $$\frac{\phi(n)}{2^{e_1 + \cdots + e_r}}\left(\frac{2^{re}-1}{2^r - 1}+1\right)$$
    and prove that this number is at most $\phi(n)/2^{r-1}.$

    \hint{Take a primitive root $g_i$ mod $p_i^{k_i}$, write $b = g_i^{c_i}$ and count the number of possible $c_i$.}
\end{enumerate}


\vspace{5pt}





\end{document}


%%% Local Variables:
%%% mode: latex
%%% TeX-master: t
%%% End:
