\documentclass[12pt,letterpaper]{article}

%\usepackage{url}
\usepackage{fancyhdr,fancybox,comment}
\usepackage{amstext}
\usepackage{amsmath,amssymb, xcolor,enumerate}
\usepackage{rotating}
%\usepackage{multicol}
%\usepackage{pictexwd,pictexplus}
\usepackage{pictexwd}



\setlength{\textwidth}{7in}
\setlength{\evensidemargin}{-0.25in}
\setlength{\oddsidemargin}{-0.25in}
\setlength{\parskip}{.3\baselineskip}
\setlength{\topmargin}{-1.0in}
%\setlength{\topmargin}{-0.2in}
%\setlength{\topmargin}{-0in}
\setlength{\textheight}{9.5in}

\newcommand{\ds}{\displaystyle}
\newcommand{\add}[1]{{\color{blue} #1}}
\newcommand{\alert}[1]{{\color{red} #1}}
\newcommand{\pnt}[1]{\textbf{[#1]}}
\newcommand{\q}{\pnt{5} }
\newcommand{\hint}[1]{\noindent{\color{purple} Hint: #1}}
\def\R{{\mathbb R}}
\def\Z{{\mathbb Z}}
\def\N{{\mathbb N}}
\def\Q{{\mathbb Q}}
\def\C{{\mathbb C}}
\def\F{{\mathbb F}}
\def\lcm{{\rm lcm}}
\def\prob#1{\vskip13pt\noindent\llap{{\bf #1:\ }}}
\font\Bigrm=cmr12 scaled\magstep1

%               Problem and Part
\newcounter{problemnumber}
\newcounter{partnumber}
\newcommand{\Problem}{\stepcounter{problemnumber}\setcounter{partnumber}{0}\item[\fbox{\parbox{.18in}{\hfil\theproblemnumber\hfil}}]}
\newcommand{\InProblem}[1]{\stepcounter{problemnumber}\setcounter{partnumber}{0}\fbox{\parbox{.18in}{\hfil\theproblemnumber\hfil}}\ \ \parbox[t]{1.9in}{#1}}
\newcommand{\InBigProblem}[1]{\stepcounter{problemnumber}\setcounter{partnumber}{0}\fbox{\parbox{.18in}{\hfil\theproblemnumber\hfil}}\ \ \parbox[t]{2.75in}{#1}}
\newcommand{\Part}{\stepcounter{partnumber}\item[(\alph{partnumber})]}
\newcommand{\InPart}[1]{\stepcounter{partnumber}(\alph{partnumber})\ \ \parbox[t]{2.25in}{#1}}
\newcommand{\InSmallPart}[1]{\stepcounter{partnumber}(\alph{partnumber})\ \ \parbox[t]{1.25in}{#1}}

%% For Answerbox:
\newcounter{answernumber}
\newcommand{\Ans}{%
  \stepcounter{answernumber}
  \fbox{\parbox{.18in}{\hfil\theanswernumber\hfil}}
}



\pagestyle{empty}

\begin{document}
%\thispagestyle{fancy}
%%
\noindent{\Bigrm HW 2 due Friday September 22}



\prob1\q A positive integer is said to be squarefree if it is not divisible by the square of any prime. 
\begin{enumerate}[(a)]
    \item \pnt{2} Prove that every positive integer can be written uniquely as $st^2$ where $s\in\N$ is squarefree and $t\in\N$.

    \hint{What should the prime factorizations of $s$ and $t$ look like?}
    \item \pnt{3} Use this (and not the prime number theorem) to prove that there is a constant $C>0$ such that $\displaystyle\pi(x) \geq C\log x$ for any real number $x\geq2$.

    \hint{How many squarefree integers can be made from the primes less than $x$? How many squares are there less than $x$?}
\end{enumerate} 
This is Erd\"{o}s' proof of the infinitude of primes.

\vspace{5pt}


\prob2\pnt{5} Let $p$ be a prime. For any positive integer $n$, we can express it in base $p$ as $n = n_0 + n_1p + \cdots + n_kp^k$ and write $s_p(n)=n_0 + n_1 + \cdots + n_k$. Prove that $$\nu_p(n!) = \frac{n - s_p(n)}{p-1}.$$

\hint{If $\nu_p(n) = \ell$, what do the base $p$ expansions of $n$ and $n-1$ look like?}

\vspace{5pt}


\prob3\pnt{5} Suppose $a,b,n$ are positive integers such that $a!b!$ divides $n!$. Prove that $\displaystyle a + b \leq n + 1 + 2\,\frac{\log n}{\log 2}$.

\hint{Use Problem 2 to show that $\nu_2(m!)$ is not too far from $m$ for any $m\in\N$.}


\vspace{5pt}


\prob4\pnt{5} Let $m,n$ be positive integers such that $1\leq m\leq n$. Prove that $\displaystyle m\binom{n}{m} \mid \lcm(1,\ldots,n).$



\vspace{5pt}


\prob5\pnt{5} Let $n$ be a positive integer. 
\begin{enumerate}[(a)]
    \item \pnt{2} Prove that if $N$ is an integer such that $n\binom{2n+1}{n}\mid N$ and $(n+1)\binom{2n+1}{n+1}\mid N$, then $$n(n+1)\binom{2n+1}{n}\mid N.$$

    \item \pnt{2} Prove that $n4^n\leq \lcm(1,\ldots,2n+1)$.

    \hint{One couldn't help but wonder why Part (a) is there.}
    \item \pnt{1} Prove that for any integer $m\geq 7$, $\lcm(1,\ldots,m)\geq 2^m$.
    
\end{enumerate} 
It then follows that for any integer $x\geq 2$, $\ds\pi(x) \geq \log 2\frac{x}{\log x}$ by checking the smaller values directly.

\vspace{5pt}


\end{document}


%%% Local Variables:
%%% mode: latex
%%% TeX-master: t
%%% End:
