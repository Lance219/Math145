\documentclass[12pt,letterpaper]{article}

%\usepackage{url}
\usepackage{fancyhdr,fancybox,comment}
\usepackage{amstext}
\usepackage{amsmath,amssymb,xcolor,enumerate}
\usepackage{rotating}
%\usepackage{multicol}
%\usepackage{pictexwd,pictexplus}
\usepackage{pictexwd}



\setlength{\textwidth}{7in}
\setlength{\evensidemargin}{-0.25in}
\setlength{\oddsidemargin}{-0.25in}
\setlength{\parskip}{.3\baselineskip}
\setlength{\topmargin}{-1.0in}
%\setlength{\topmargin}{-0.2in}
%\setlength{\topmargin}{-0in}
\setlength{\textheight}{9.5in}

\newcommand{\ds}{\displaystyle}
\newcommand{\add}[1]{{\color{blue} #1}}
\newcommand{\alert}[1]{{\color{red} #1}}
\newcommand{\pnt}[1]{\textbf{[#1]}}
\newcommand{\q}{\pnt{5} }
\newcommand{\hint}[1]{\noindent{\color{purple} Hint: #1}}
\renewcommand{\mark}[1]{\noindent{\color{red} #1}}
\newcommand{\tot}[2]{\noindent\textsc{\footnotesize This homework is marked out of \textbf{#1} and graded out of \textbf{#2}}.}
\def\R{{\mathbb R}}
\def\Z{{\mathbb Z}}
\def\N{{\mathbb N}}
\def\Q{{\mathbb Q}}
\def\C{{\mathbb C}}
\def\F{{\mathbb F}}
\def\lcm{{\rm lcm}}
\def\prob#1{\vskip13pt\noindent\llap{{\bf #1:\ }}}
\font\Bigrm=cmr12 scaled\magstep1

%               Problem and Part
\newcounter{problemnumber}
\newcounter{partnumber}
\newcommand{\Problem}{\stepcounter{problemnumber}\setcounter{partnumber}{0}\item[\fbox{\parbox{.18in}{\hfil\theproblemnumber\hfil}}]}
\newcommand{\InProblem}[1]{\stepcounter{problemnumber}\setcounter{partnumber}{0}\fbox{\parbox{.18in}{\hfil\theproblemnumber\hfil}}\ \ \parbox[t]{1.9in}{#1}}
\newcommand{\InBigProblem}[1]{\stepcounter{problemnumber}\setcounter{partnumber}{0}\fbox{\parbox{.18in}{\hfil\theproblemnumber\hfil}}\ \ \parbox[t]{2.75in}{#1}}
\newcommand{\Part}{\stepcounter{partnumber}\item[(\alph{partnumber})]}
\newcommand{\InPart}[1]{\stepcounter{partnumber}(\alph{partnumber})\ \ \parbox[t]{2.25in}{#1}}
\newcommand{\InSmallPart}[1]{\stepcounter{partnumber}(\alph{partnumber})\ \ \parbox[t]{1.25in}{#1}}

%% For Answerbox:
\newcounter{answernumber}
\newcommand{\Ans}{%
  \stepcounter{answernumber}
  \fbox{\parbox{.18in}{\hfil\theanswernumber\hfil}}
}



\pagestyle{empty}

\begin{document}
%\thispagestyle{fancy}
%%
\noindent{\Bigrm HW 5 due Friday October 20}

%\tot{25}{20}

\prob1\pnt{4} Let $F$ be a field. 
\begin{enumerate}[(a)]
        \item \pnt{3} Suppose $f(x)\in F[x]$ is irreducible. If $f(x)\mid a(x)b(x)$ in $F[x]$, then $f(x)\mid a(x)$ or $f(x)\mid b(x)$.

        \hint{Recall the proof of Euclid's lemma for primes and try mimicing that. Whenever you want to talk about $\gcd(f(x),g(x))$, use the ideal $(f(x),g(x))$ instead.}
        \item \pnt{1} Every non-constant polynomial in $F[x]$ can be factored into a product of irreducible polynomials in $F[x]$.

        \hint{No need to use the $p$-adic valuation method. An induction on the degree of the polynomial is enough.}
    \end{enumerate}

\vspace{5pt}

\prob2\pnt{3} Prove that $n\mid\phi(a^n - 1)$ for any positive integers $a,n$ such that $a^n - 1 > 1$.

\hint{What is $o_{a^n-1}(a)?$}

\vspace{5pt}

\prob3\q Let $R$ be a finite commutative ring with $p^2$ elements, where $p$ is a prime. Suppose $R$ is not a field. Prove that $R\cong\Z/p^2\Z$ or $\F_p[x]/(x^2)$ or $\F_p\times\F_p$.

\hint{If $R$ has characteristic $p$, then the prime subring of $R$ is $\F_p$. Take some $\alpha\in R\backslash\F_p$ and consider the evaluation map $\text{ev}_\alpha:\F_p[x]\rightarrow R$.}

\vspace{5pt}

\prob4\q Let $R$ be a finite commutative ring with $n$ elements. Let $p\mid n$ be a prime. Let $R^{\F_p}$ denote the set of functions from $\F_p$ to $R$. For any $r\in R$, define the set $$S(r) = \{f\in R^{\F_p}\colon \sum_{a\in\F_p}f(a) = f(0) + f(1) + \cdots + f(p-1) = r\}.$$
For $f,g\in S(r)$, we say $f\sim g$ if and only if there exists $a\in\F_p$ such that $$f(x) = g(x+a)\qquad\mbox{for every }x\in\F_p.$$
\begin{enumerate}[(a)]
    \item \pnt{1} Prove that $\sim$ defines an equivalence relation on $S(r)$. (See the paragraph after the proof of Corollary 8.2 for the definition of an equivalence relation.)
    \item \pnt{2} Prove that each equivalence class in $S(r)$ has size $p$ or $1$.

    \hint{For any $f\in S(r)$, the equivalence class $[f]$ consists of all functions of the from $h(x)= f(x + a)$ for some $a\in \F_p$.}
    \item \pnt{2} Prove that $R$ has an element $a$ with $o_+(a) = p$.

    \hint{Consider $S(0)$. What do equivalence classes of size $1$ look like? (You may use the fact that distinct equivalence classes are disjoint without proof.)}
\end{enumerate}

\vspace{5pt}

\prob5\pnt{8} Let $R$ be a finite commutative ring with $n$ elements. Let $n = p_1^{a_1}\cdots p_r^{a_r}$ be the prime factorization of $n$ (so that $p_1,\ldots,p_r$ are distinct primes and $a_1,\ldots,a_r\geq 1$). For any prime $p\mid n$, let $$R[p^\infty]=\{\alpha\in R\colon \exists k\in\N, p^k\alpha = 0\}.$$ 
\begin{enumerate}[(a)]
    \item \pnt{2} Prove that there exist $e_1,\ldots,e_r\in R$ such that $e_i\in R[p_i^\infty]$ for any $i=1,\ldots,r$ and $1=e_1+\cdots+e_r$.

    \hint{We don't have any real information about $R$. Try to find integers $q_i$ so that $\gcd(q_1,\ldots,q_r) = 1$ and $q_i\in R[p_i^\infty]$ when viewed as elements of $R$. Then take $e_i$ to be some multiple of $q_i$.}
    \item \pnt{2} Prove that for any $i =1,\ldots,r$, the above $e_i$ is an idempotent, that is $e_i^2 = e_i$.

    \hint{What does $ab$ equal to if $a\in R[p_i^\infty]$ and $b\in R[p_j^\infty]$ for $i\neq j$? Use this to compute $e_ie_j$ for $i\neq j$.}
    \item \pnt{2} Prove that for any $i =1,\ldots,r$, $|R/(e_i-1)|$ is a power of $p_i$.

    \hint{In $R/(e_i-1)$, we have $r+(e_i-1) = re_i + (e_i-1)$. Prove that $p_i^{a_i}\alpha = 0$ in $R/(e_i-1)$ for every $\alpha\in R/(e_i-1)$. If $|R/(e_i-1)|$ has another prime divisor $q$, use Problem 4c) to find an element with additive order $q$.}
    \item \pnt{2} Prove that $\displaystyle R\cong R/(e_1 - 1) \times R/(e_2-1)\times\cdots\times R/(e_r -1).$

    \hint{Take the most natural map and prove that it works.}
\end{enumerate}
This implies that the size of a finite field can only have one prime divisor, and reduces the classification of finite commutative rings to the case of $p$-rings, which are rings whose sizes are powers of a prime $p$.






\end{document}


%%% Local Variables:
%%% mode: latex
%%% TeX-master: t
%%% End:
