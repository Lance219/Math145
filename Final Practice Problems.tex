\documentclass[12pt,letterpaper]{article}

%\usepackage{url}
\usepackage{fancyhdr,fancybox,comment}
\usepackage{amstext}
\usepackage{amsmath,amssymb,color,enumerate}
\usepackage{rotating}
%\usepackage{multicol}
%\usepackage{pictexwd,pictexplus}
\usepackage{pictexwd}



\setlength{\textwidth}{7in}
\setlength{\evensidemargin}{-0.25in}
\setlength{\oddsidemargin}{-0.25in}
\setlength{\parskip}{.3\baselineskip}
\setlength{\topmargin}{-1.0in}
%\setlength{\topmargin}{-0.2in}
%\setlength{\topmargin}{-0in}
\setlength{\textheight}{9.5in}

\newcommand{\ds}{\displaystyle}
\newcommand{\add}[1]{{\color{blue} #1}}
\newcommand{\alert}[1]{{\color{red} #1}}
\def\R{{\mathbb R}}
\def\Z{{\mathbb Z}}
\def\N{{\mathbb N}}
\def\Q{{\mathbb Q}}
\def\C{{\mathbb C}}
\def\F{{\mathbb F}}
\def\lcm{{\rm lcm}}
\def\prob#1{\vskip13pt\noindent\llap{{\bf #1:\ }}}
\font\Bigrm=cmr12 scaled\magstep1

%               Problem and Part
\newcounter{problemnumber}
\newcounter{partnumber}
\newcommand{\answerbox}{\\[2mm]
\fbox{ \begin{minipage}{18mm} \hfill\vspace{9mm} \end{minipage}}\\[5mm]}
\newcommand{\Problem}{\stepcounter{problemnumber}\setcounter{partnumber}{0}\item[\fbox{\parbox{.18in}{\hfil\theproblemnumber\hfil}}]}
\newcommand{\InProblem}[1]{\stepcounter{problemnumber}\setcounter{partnumber}{0}\fbox{\parbox{.18in}{\hfil\theproblemnumber\hfil}}\ \ \parbox[t]{1.9in}{#1}}
\newcommand{\InBigProblem}[1]{\stepcounter{problemnumber}\setcounter{partnumber}{0}\fbox{\parbox{.18in}{\hfil\theproblemnumber\hfil}}\ \ \parbox[t]{2.75in}{#1}}
\newcommand{\Part}{\stepcounter{partnumber}\item[(\alph{partnumber})]}
\newcommand{\InPart}[1]{\stepcounter{partnumber}(\alph{partnumber})\ \ \parbox[t]{2.25in}{#1}}
\newcommand{\InSmallPart}[1]{\stepcounter{partnumber}(\alph{partnumber})\ \ \parbox[t]{1.25in}{#1}}

%% For Answerbox:
\newcounter{answernumber}
\newcommand{\Ans}{%
  \stepcounter{answernumber}
  \fbox{\parbox{.18in}{\hfil\theanswernumber\hfil}}
}



\pagestyle{empty}

\begin{document}
%\thispagestyle{fancy}
%%
\noindent{\Bigrm Final Exam Practice Problems}
\vskip15pt

Determine if the following statements are True or False.

\begin{enumerate}
\item (*) There exists a positive integer $m$ such that $\Phi_m(69) = 420.$
\item (**) Every positive rational number can be expressed as a quotient of products of numbers of the form $\binom{n}{\lfloor n/2\rfloor}$ where $n$ is a positive integer.
\item $69$ is a quadratic residue mod $420$.
\item (**) The additive group $(\Q,+)$ and the multiplicative group $(\Q^\times,\cdot)$ are isomorphic. \alert{Groups like these won't appear on the exam.}
\item (**) $x^4+x+1$ is irreducible in $\mathbb Q[x].$
\item (*) The sum $\ds\sum_{n=1}^\infty \frac{1}{\text{lcm}(1,2,\ldots,n)}$ is finite. \alert{There will be no infinite sum on the exam.}
\item If $d>1$ is an integer such that $(d-1)!\equiv -1\pmod{d}$, then $d$ is a prime.
\end{enumerate}
\end{document}